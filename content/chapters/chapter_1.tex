\section{Hinweise}
\label{sec:hinweise}

Folgendes Dokument mit Hinweisen soll als Vorlage für die Seminararbeit gelten.
Nutzen Sie dieses gerne zur Anfertigung ihrer Arbeit.
Die Vorlage ist im \LaTeX{} und Word Format vorhanden.
Insofern die Arbeit den vorgegebenen Formalien entspricht, steht es Ihnen frei, diese in einem Editor/ Format Ihrer Wahl zu erstellen.

\subsection{Allgemein}
Der Umfang der Arbeit soll ohne Verzeichnisse und Titelblatt mindestens 25 Seiten und maximal 35 Seiten enthalten.
Die erste Seite der Arbeit ist das Deckblatt, darauffolgend das Inhaltsverzeichnis, gefolgt von, wenn benötigt, dem Abbildungsverzeichnis, dem Tabellenverzeichnis und dem Listingverzeichnis.
Anschließend wird der Text ausformuliert und darauffolgend die Quellen aufgelistet.
Die letzte Seite ist die eidesstattliche Erklärung.

\subsection{Formatierung}
Die Seitenränder sind auf A4 oben in der Breite 3cm, rechts 2,5cm, unten 3,5cm und links 3cm zu halten.
Die zu verwendende Schriftart sollte Times New Roman oder eine ähnliche Serife Schriftart in der Größe 12pt sein.
Der Zeilenabstand ist 1,5-fach zu halten.

\subsection{Überschriften}
Die Überschriften sind numerisch fortlaufend zu nummerieren, wobei nicht mehr als drei Ebenen verwendet werden sollen.

\subsection{Verweise}
Verweise im Text sollen in Fußnoten erstellt werden. 
Im Fußnotentext steht ein Vollverweis auf die Quelle. 
Verweise, die einen einzelnen Satz betreffen, stehen direkt hinter dem Satz und vor dem Punkt\footcite{example}.
Verweise, die im Absatz aufgegriffen werden, kommen an das Ende des Absatzes hinter den letzten Punkt.\footcite{example}

\subsection{Quellen}
Die Verwendeten Quellen müssen mit dem im Text referenzierten Inhalt übereinstimmen.
Bei Zweifel sind die Quellen mit nachzuweisen.

\subsection{Prüfung}
Die Seminararbeiten werden mit diversen Tools zur Prüfung des Eigenanteils geprüft.
Darunter fallen auch Tests, die den Einsatz von KI-generierten Texten erkennen.
Diese sind \emph{originaliti.ai} und \emph{turnitin}.